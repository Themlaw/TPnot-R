% Options for packages loaded elsewhere
\PassOptionsToPackage{unicode}{hyperref}
\PassOptionsToPackage{hyphens}{url}
%
\documentclass[
]{article}
\usepackage{amsmath,amssymb}
\usepackage{iftex}
\ifPDFTeX
  \usepackage[T1]{fontenc}
  \usepackage[utf8]{inputenc}
  \usepackage{textcomp} % provide euro and other symbols
\else % if luatex or xetex
  \usepackage{unicode-math} % this also loads fontspec
  \defaultfontfeatures{Scale=MatchLowercase}
  \defaultfontfeatures[\rmfamily]{Ligatures=TeX,Scale=1}
\fi
\usepackage{lmodern}
\ifPDFTeX\else
  % xetex/luatex font selection
\fi
% Use upquote if available, for straight quotes in verbatim environments
\IfFileExists{upquote.sty}{\usepackage{upquote}}{}
\IfFileExists{microtype.sty}{% use microtype if available
  \usepackage[]{microtype}
  \UseMicrotypeSet[protrusion]{basicmath} % disable protrusion for tt fonts
}{}
\makeatletter
\@ifundefined{KOMAClassName}{% if non-KOMA class
  \IfFileExists{parskip.sty}{%
    \usepackage{parskip}
  }{% else
    \setlength{\parindent}{0pt}
    \setlength{\parskip}{6pt plus 2pt minus 1pt}}
}{% if KOMA class
  \KOMAoptions{parskip=half}}
\makeatother
\usepackage{xcolor}
\usepackage[margin=1in]{geometry}
\usepackage{color}
\usepackage{fancyvrb}
\newcommand{\VerbBar}{|}
\newcommand{\VERB}{\Verb[commandchars=\\\{\}]}
\DefineVerbatimEnvironment{Highlighting}{Verbatim}{commandchars=\\\{\}}
% Add ',fontsize=\small' for more characters per line
\usepackage{framed}
\definecolor{shadecolor}{RGB}{248,248,248}
\newenvironment{Shaded}{\begin{snugshade}}{\end{snugshade}}
\newcommand{\AlertTok}[1]{\textcolor[rgb]{0.94,0.16,0.16}{#1}}
\newcommand{\AnnotationTok}[1]{\textcolor[rgb]{0.56,0.35,0.01}{\textbf{\textit{#1}}}}
\newcommand{\AttributeTok}[1]{\textcolor[rgb]{0.13,0.29,0.53}{#1}}
\newcommand{\BaseNTok}[1]{\textcolor[rgb]{0.00,0.00,0.81}{#1}}
\newcommand{\BuiltInTok}[1]{#1}
\newcommand{\CharTok}[1]{\textcolor[rgb]{0.31,0.60,0.02}{#1}}
\newcommand{\CommentTok}[1]{\textcolor[rgb]{0.56,0.35,0.01}{\textit{#1}}}
\newcommand{\CommentVarTok}[1]{\textcolor[rgb]{0.56,0.35,0.01}{\textbf{\textit{#1}}}}
\newcommand{\ConstantTok}[1]{\textcolor[rgb]{0.56,0.35,0.01}{#1}}
\newcommand{\ControlFlowTok}[1]{\textcolor[rgb]{0.13,0.29,0.53}{\textbf{#1}}}
\newcommand{\DataTypeTok}[1]{\textcolor[rgb]{0.13,0.29,0.53}{#1}}
\newcommand{\DecValTok}[1]{\textcolor[rgb]{0.00,0.00,0.81}{#1}}
\newcommand{\DocumentationTok}[1]{\textcolor[rgb]{0.56,0.35,0.01}{\textbf{\textit{#1}}}}
\newcommand{\ErrorTok}[1]{\textcolor[rgb]{0.64,0.00,0.00}{\textbf{#1}}}
\newcommand{\ExtensionTok}[1]{#1}
\newcommand{\FloatTok}[1]{\textcolor[rgb]{0.00,0.00,0.81}{#1}}
\newcommand{\FunctionTok}[1]{\textcolor[rgb]{0.13,0.29,0.53}{\textbf{#1}}}
\newcommand{\ImportTok}[1]{#1}
\newcommand{\InformationTok}[1]{\textcolor[rgb]{0.56,0.35,0.01}{\textbf{\textit{#1}}}}
\newcommand{\KeywordTok}[1]{\textcolor[rgb]{0.13,0.29,0.53}{\textbf{#1}}}
\newcommand{\NormalTok}[1]{#1}
\newcommand{\OperatorTok}[1]{\textcolor[rgb]{0.81,0.36,0.00}{\textbf{#1}}}
\newcommand{\OtherTok}[1]{\textcolor[rgb]{0.56,0.35,0.01}{#1}}
\newcommand{\PreprocessorTok}[1]{\textcolor[rgb]{0.56,0.35,0.01}{\textit{#1}}}
\newcommand{\RegionMarkerTok}[1]{#1}
\newcommand{\SpecialCharTok}[1]{\textcolor[rgb]{0.81,0.36,0.00}{\textbf{#1}}}
\newcommand{\SpecialStringTok}[1]{\textcolor[rgb]{0.31,0.60,0.02}{#1}}
\newcommand{\StringTok}[1]{\textcolor[rgb]{0.31,0.60,0.02}{#1}}
\newcommand{\VariableTok}[1]{\textcolor[rgb]{0.00,0.00,0.00}{#1}}
\newcommand{\VerbatimStringTok}[1]{\textcolor[rgb]{0.31,0.60,0.02}{#1}}
\newcommand{\WarningTok}[1]{\textcolor[rgb]{0.56,0.35,0.01}{\textbf{\textit{#1}}}}
\usepackage{graphicx}
\makeatletter
\def\maxwidth{\ifdim\Gin@nat@width>\linewidth\linewidth\else\Gin@nat@width\fi}
\def\maxheight{\ifdim\Gin@nat@height>\textheight\textheight\else\Gin@nat@height\fi}
\makeatother
% Scale images if necessary, so that they will not overflow the page
% margins by default, and it is still possible to overwrite the defaults
% using explicit options in \includegraphics[width, height, ...]{}
\setkeys{Gin}{width=\maxwidth,height=\maxheight,keepaspectratio}
% Set default figure placement to htbp
\makeatletter
\def\fps@figure{htbp}
\makeatother
\setlength{\emergencystretch}{3em} % prevent overfull lines
\providecommand{\tightlist}{%
  \setlength{\itemsep}{0pt}\setlength{\parskip}{0pt}}
\setcounter{secnumdepth}{-\maxdimen} % remove section numbering
\ifLuaTeX
  \usepackage{selnolig}  % disable illegal ligatures
\fi
\usepackage{bookmark}
\IfFileExists{xurl.sty}{\usepackage{xurl}}{} % add URL line breaks if available
\urlstyle{same}
\hypersetup{
  pdftitle={Rendu\_titouan},
  hidelinks,
  pdfcreator={LaTeX via pandoc}}

\title{Rendu\_titouan}
\author{}
\date{\vspace{-2.5em}}

\begin{document}
\maketitle

nom1\_nom2.Rmd

nom1\_nom2.html ou et .pdf

\begin{Shaded}
\begin{Highlighting}[]
\NormalTok{knitr}\SpecialCharTok{::}\NormalTok{opts\_chunk}\SpecialCharTok{$}\FunctionTok{set}\NormalTok{(}\AttributeTok{echo =} \ConstantTok{TRUE}\NormalTok{)}
\FunctionTok{library}\NormalTok{(ggplot2)}
\FunctionTok{library}\NormalTok{(factoextra)}
\end{Highlighting}
\end{Shaded}

\begin{verbatim}
## Warning: le package 'factoextra' a été compilé avec la version R 4.4.3
\end{verbatim}

\begin{verbatim}
## Welcome! Want to learn more? See two factoextra-related books at https://goo.gl/ve3WBa
\end{verbatim}

\begin{Shaded}
\begin{Highlighting}[]
\FunctionTok{library}\NormalTok{(dplyr)}
\end{Highlighting}
\end{Shaded}

\begin{verbatim}
## 
## Attachement du package : 'dplyr'
\end{verbatim}

\begin{verbatim}
## Les objets suivants sont masqués depuis 'package:stats':
## 
##     filter, lag
\end{verbatim}

\begin{verbatim}
## Les objets suivants sont masqués depuis 'package:base':
## 
##     intersect, setdiff, setequal, union
\end{verbatim}

\begin{Shaded}
\begin{Highlighting}[]
\NormalTok{Data}\OtherTok{=}\FunctionTok{read.csv}\NormalTok{(}\AttributeTok{file=}\StringTok{"./abalone/abalone.data"}\NormalTok{)}
\NormalTok{Data }\OtherTok{=}\NormalTok{ Data[}\DecValTok{1}\SpecialCharTok{:}\DecValTok{100}\NormalTok{,}\DecValTok{2}\SpecialCharTok{:}\DecValTok{9}\NormalTok{]}
\CommentTok{\#head(Data)}
\NormalTok{Data }\OtherTok{=} \FunctionTok{rename}\NormalTok{(Data, }\StringTok{"Longueur"}\OtherTok{=}\StringTok{\textasciigrave{}}\AttributeTok{X0.455}\StringTok{\textasciigrave{}}\NormalTok{, }\StringTok{"Diamètre"} \OtherTok{=} \StringTok{\textasciigrave{}}\AttributeTok{X0.365}\StringTok{\textasciigrave{}}\NormalTok{, }\StringTok{"Hauteur"} \OtherTok{=} \StringTok{\textasciigrave{}}\AttributeTok{X0.095}\StringTok{\textasciigrave{}}\NormalTok{, }\StringTok{"Poids\_Total"}\OtherTok{=}\StringTok{\textasciigrave{}}\AttributeTok{X0.514}\StringTok{\textasciigrave{}}\NormalTok{, }\StringTok{"Poids\_Décortiqué"}\OtherTok{=}\StringTok{\textasciigrave{}}\AttributeTok{X0.2245}\StringTok{\textasciigrave{}}\NormalTok{,}\StringTok{"Poids\_Viscères"}\OtherTok{=}\StringTok{\textasciigrave{}}\AttributeTok{X0.101}\StringTok{\textasciigrave{}}\NormalTok{,}\StringTok{"Poids\_Coquille"}\OtherTok{=}\StringTok{\textasciigrave{}}\AttributeTok{X0.15}\StringTok{\textasciigrave{}}\NormalTok{, }\StringTok{"Anneaux"}\OtherTok{=}\StringTok{\textasciigrave{}}\AttributeTok{X15}\StringTok{\textasciigrave{}}\NormalTok{)}
\FunctionTok{head}\NormalTok{(Data)}
\end{Highlighting}
\end{Shaded}

\begin{verbatim}
##   Longueur Diamètre Hauteur Poids_Total Poids_Décortiqué Poids_Viscères
## 1    0.350    0.265   0.090      0.2255           0.0995         0.0485
## 2    0.530    0.420   0.135      0.6770           0.2565         0.1415
## 3    0.440    0.365   0.125      0.5160           0.2155         0.1140
## 4    0.330    0.255   0.080      0.2050           0.0895         0.0395
## 5    0.425    0.300   0.095      0.3515           0.1410         0.0775
## 6    0.530    0.415   0.150      0.7775           0.2370         0.1415
##   Poids_Coquille Anneaux
## 1          0.070       7
## 2          0.210       9
## 3          0.155      10
## 4          0.055       7
## 5          0.120       8
## 6          0.330      20
\end{verbatim}

\#Question 1 On centre dans un premier temps les variables de notre jeu
de données. Pour cela, on commence par calculer les moyennes de chaque
colonne, que l'on met dans un vecteur ligne, on répète cette ligne n
fois pour obtenir une matrice de la même taille que notre jeu de
données, et on soustrait cette matrice à notre jeu de données initial.

\begin{Shaded}
\begin{Highlighting}[]
\NormalTok{moyennes }\OtherTok{=} \FunctionTok{colMeans}\NormalTok{(Data)}
\NormalTok{n }\OtherTok{=} \FunctionTok{nrow}\NormalTok{(Data)}
\NormalTok{p }\OtherTok{=} \FunctionTok{ncol}\NormalTok{(Data)}
\NormalTok{moyennes }\OtherTok{=} \FunctionTok{matrix}\NormalTok{(}\FunctionTok{rep}\NormalTok{(moyennes, }\AttributeTok{each=}\NormalTok{n))}
\NormalTok{Data\_centre }\OtherTok{=}\NormalTok{ Data }\SpecialCharTok{{-}}\NormalTok{ moyennes}
\FunctionTok{head}\NormalTok{(Data\_centre)}
\end{Highlighting}
\end{Shaded}

\begin{verbatim}
##   Longueur Diamètre Hauteur Poids_Total Poids_Décortiqué Poids_Viscères
## 1 -0.15305  -0.1288 -0.0413    -0.50182         -0.19025       -0.11183
## 2  0.02695   0.0262  0.0037    -0.05032         -0.03325       -0.01883
## 3 -0.06305  -0.0288 -0.0063    -0.21132         -0.07425       -0.04633
## 4 -0.17305  -0.1388 -0.0513    -0.52232         -0.20025       -0.12083
## 5 -0.07805  -0.0938 -0.0363    -0.37582         -0.14875       -0.08283
## 6  0.02695   0.0212  0.0187     0.05018         -0.05275       -0.01883
##   Poids_Coquille Anneaux
## 1       -0.16012   -3.58
## 2       -0.02012   -1.58
## 3       -0.07512   -0.58
## 4       -0.17512   -3.58
## 5       -0.11012   -2.58
## 6        0.09988    9.42
\end{verbatim}

\#Question 2 On cherche ici à réduire notre data set, pour cela on
calcule l'écart type des données de chaque colonne (argument 2), en
appliquant la fonction sd à chaque colonne de notre jeu de données. On
répète ensuite cette ligne n fois pour obtenir une matrice de la même
taille que notre jeu de données, et on divise notre jeu de données
initial par cette matrice.

\begin{Shaded}
\begin{Highlighting}[]
\NormalTok{ecart\_type }\OtherTok{=} \FunctionTok{apply}\NormalTok{(Data\_centre, }\DecValTok{2}\NormalTok{, sd)}
\NormalTok{ecart\_type }\OtherTok{=} \FunctionTok{matrix}\NormalTok{(}\FunctionTok{rep}\NormalTok{(ecart\_type, }\AttributeTok{each=}\NormalTok{n))}
\NormalTok{Data\_centre\_reduit }\OtherTok{=}\NormalTok{ Data\_centre}\SpecialCharTok{/}\NormalTok{ecart\_type}
\FunctionTok{head}\NormalTok{(Data\_centre\_reduit)}
\end{Highlighting}
\end{Shaded}

\begin{verbatim}
##     Longueur   Diamètre    Hauteur Poids_Total Poids_Décortiqué Poids_Viscères
## 1 -1.4570720 -1.4797441 -1.2166369  -1.2539831       -1.2472569     -1.2425915
## 2  0.2565703  0.3010038  0.1089965  -0.1257431       -0.2179831     -0.2092283
## 3 -0.6002508 -0.3308745 -0.1855887  -0.5280613       -0.4867744     -0.5147927
## 4 -1.6474767 -1.5946311 -1.5112221  -1.3052099       -1.3128157     -1.3425944
## 5 -0.7430544 -1.0776397 -1.0693443  -0.9391254       -0.9751877     -0.9203600
## 6  0.2565703  0.2435604  0.5508743   0.1253933       -0.3458229     -0.2092283
##   Poids_Coquille    Anneaux
## 1     -1.2091822 -1.0081629
## 2     -0.1519407 -0.4449434
## 3     -0.5672856 -0.1633337
## 4     -1.3224581 -1.0081629
## 5     -0.8315960 -0.7265531
## 6      0.7542663  2.6527638
\end{verbatim}

les données sont alors de l'ordre de grandeur de 1 (entre -2 et 2
environ)

\#Question 3 On vient ensuite stocker la matrice de covariance normé
dans hatsigma

\begin{Shaded}
\begin{Highlighting}[]
\NormalTok{hatSigma }\OtherTok{\textless{}{-}} \FunctionTok{cov}\NormalTok{(Data\_centre\_reduit)}
\NormalTok{hatSigma}
\end{Highlighting}
\end{Shaded}

\begin{verbatim}
##                   Longueur  Diamètre   Hauteur Poids_Total Poids_Décortiqué
## Longueur         1.0000000 0.9872936 0.9258558   0.9224579        0.9161325
## Diamètre         0.9872936 1.0000000 0.9362003   0.9399541        0.9279551
## Hauteur          0.9258558 0.9362003 1.0000000   0.9084448        0.8803453
## Poids_Total      0.9224579 0.9399541 0.9084448   1.0000000        0.9692632
## Poids_Décortiqué 0.9161325 0.9279551 0.8803453   0.9692632        1.0000000
## Poids_Viscères   0.8936023 0.9001539 0.8863588   0.9355658        0.9515333
## Poids_Coquille   0.8758735 0.8975438 0.8790295   0.9603543        0.8796758
## Anneaux          0.7159627 0.7182953 0.7000823   0.7411373        0.7068338
##                  Poids_Viscères Poids_Coquille   Anneaux
## Longueur              0.8936023      0.8758735 0.7159627
## Diamètre              0.9001539      0.8975438 0.7182953
## Hauteur               0.8863588      0.8790295 0.7000823
## Poids_Total           0.9355658      0.9603543 0.7411373
## Poids_Décortiqué      0.9515333      0.8796758 0.7068338
## Poids_Viscères        1.0000000      0.8351496 0.6577045
## Poids_Coquille        0.8351496      1.0000000 0.7569556
## Anneaux               0.6577045      0.7569556 1.0000000
\end{verbatim}

\#Question 4 On vient diagonaliser la matrice

\begin{Shaded}
\begin{Highlighting}[]
\NormalTok{eigen\_hatSigma }\OtherTok{\textless{}{-}} \FunctionTok{eigen}\NormalTok{(hatSigma)}
\NormalTok{eigen\_hatSigma}\SpecialCharTok{$}\NormalTok{values    }\CommentTok{\# Valeurs propres}
\end{Highlighting}
\end{Shaded}

\begin{verbatim}
## [1] 7.076876406 0.435328067 0.180050066 0.161642005 0.094374764 0.036070543
## [7] 0.010864343 0.004793806
\end{verbatim}

\begin{Shaded}
\begin{Highlighting}[]
\NormalTok{eigen\_hatSigma}\SpecialCharTok{$}\NormalTok{vectors   }\CommentTok{\# Vecteurs propres}
\end{Highlighting}
\end{Shaded}

\begin{verbatim}
##           [,1]        [,2]        [,3]        [,4]        [,5]        [,6]
## [1,] 0.3627168 -0.12277760  0.41775891  0.21409616 -0.43269576 -0.20416086
## [2,] 0.3662748 -0.12621440  0.36386408  0.09899002 -0.35463565 -0.04406468
## [3,] 0.3565726 -0.12019210  0.46462000  0.03504811  0.73867827  0.30395057
## [4,] 0.3695732 -0.06467148 -0.28038214 -0.29027419 -0.04964468  0.09017463
## [5,] 0.3625005 -0.16174489 -0.43255303  0.14913764 -0.21424589  0.67328380
## [6,] 0.3541017 -0.27113924 -0.45487500  0.36962359  0.30256809 -0.59008548
## [7,] 0.3544739  0.11426547 -0.03165606 -0.79473665  0.00297659 -0.23270721
## [8,] 0.2968240  0.91522268 -0.05982507  0.26154607  0.03794924 -0.01395126
##             [,7]        [,8]
## [1,]  0.63362997 -0.05107141
## [2,] -0.76033466  0.04358536
## [3,]  0.05573132 -0.01312141
## [4,] -0.01273690 -0.82801069
## [5,]  0.09505427  0.35332720
## [6,] -0.05389864  0.12210347
## [7,]  0.06889179  0.41203835
## [8,] -0.02120392 -0.01389927
\end{verbatim}

\#Question 5 On sait que l'inertie totale de données centrée réduite est
égale à la somme des valeurs propres de la matrice de covariance normée.
Et cela vaut le nombre de variables, soit, ici, 8.

On calcule alors de 2 manière: - D'abord avec la formule de référence :
Calculer la moyenne des carrés des observations pour chaque variable. -
Ensuite en sommant les valeurs propres de la matrice de covariance
normée.

\begin{Shaded}
\begin{Highlighting}[]
\CommentTok{\#Avec la formule}
\NormalTok{inertie\_total1 }\OtherTok{\textless{}{-}} \FunctionTok{sum}\NormalTok{(}\FunctionTok{apply}\NormalTok{(Data\_centre\_reduit, }\DecValTok{2}\NormalTok{, }\ControlFlowTok{function}\NormalTok{(x) }\FunctionTok{sum}\NormalTok{(x}\SpecialCharTok{\^{}}\DecValTok{2}\NormalTok{))) }\SpecialCharTok{/}\NormalTok{ (n }\SpecialCharTok{{-}} \DecValTok{1}\NormalTok{)}
\CommentTok{\#Avec les valeurs propres}
\NormalTok{inertie\_total2 }\OtherTok{\textless{}{-}} \FunctionTok{sum}\NormalTok{(eigen\_hatSigma}\SpecialCharTok{$}\NormalTok{values)}

\NormalTok{inertie\_total1}
\end{Highlighting}
\end{Shaded}

\begin{verbatim}
## [1] 8
\end{verbatim}

\begin{Shaded}
\begin{Highlighting}[]
\NormalTok{inertie\_total2}
\end{Highlighting}
\end{Shaded}

\begin{verbatim}
## [1] 8
\end{verbatim}

On trouve bien comme attendu 8 dans les deux cas

\#Question 6 On trace, en fonction de i, le pourcentage de l'inertie
expliqué par les i premières valeurs propres ainsi que celui expliqué
par la i-ème valeur propre

\begin{Shaded}
\begin{Highlighting}[]
\NormalTok{pourcentages }\OtherTok{\textless{}{-}}\NormalTok{ eigen\_hatSigma}\SpecialCharTok{$}\NormalTok{values }\SpecialCharTok{/}\NormalTok{ inertie\_total1 }\SpecialCharTok{*} \DecValTok{100}
\NormalTok{pourcentage\_cumule }\OtherTok{\textless{}{-}} \FunctionTok{cumsum}\NormalTok{(pourcentages)}

\CommentTok{\#On définit la fenêtre pour avoir 2 graphiques}
\FunctionTok{par}\NormalTok{(}\AttributeTok{mfrow =} \FunctionTok{c}\NormalTok{(}\DecValTok{1}\NormalTok{, }\DecValTok{2}\NormalTok{))}

\CommentTok{\# Graphe 1 : inertie de la i{-}ème composante}
\FunctionTok{plot}\NormalTok{(}\DecValTok{1}\SpecialCharTok{:}\FunctionTok{length}\NormalTok{(pourcentages), pourcentages, }\AttributeTok{type=}\StringTok{"b"}\NormalTok{, }\AttributeTok{pch=}\DecValTok{19}\NormalTok{, }\AttributeTok{col=}\StringTok{"red"}\NormalTok{,}
     \AttributeTok{xlab=}\StringTok{"i (nombre de composantes)"}\NormalTok{, }\AttributeTok{ylab=}\StringTok{"Pourcentage d\textquotesingle{}inertie"}\NormalTok{,}
     \AttributeTok{main=}\StringTok{"Inertie i{-}ème"}\NormalTok{)}

\CommentTok{\# Graphe 2 : inertie cumulée}
\FunctionTok{plot}\NormalTok{(}\DecValTok{1}\SpecialCharTok{:}\FunctionTok{length}\NormalTok{(pourcentages), pourcentage\_cumule, }\AttributeTok{type=}\StringTok{"b"}\NormalTok{, }\AttributeTok{pch=}\DecValTok{19}\NormalTok{, }\AttributeTok{col=}\StringTok{"blue"}\NormalTok{,}
     \AttributeTok{xlab=}\StringTok{"i (nombre de composantes)"}\NormalTok{, }\AttributeTok{ylab=}\StringTok{"Pourcentage d\textquotesingle{}inertie cumulée"}\NormalTok{,}
     \AttributeTok{main=}\StringTok{"Inertie cumulée"}\NormalTok{)}
\end{Highlighting}
\end{Shaded}

\includegraphics{Rendu_titouan_files/figure-latex/pourcentage_inertie-1.pdf}

\#Question 7 On réalise une analyse en composantes principales (ACP) sur
le jeu de données préalablement centré et réduit. Le résultat de l'ACP
est stocké dans l'objet acp\_result, qui contient notamment les scores
des individus sur les différentes composantes principales. Ensuite, on
trace la projection des individus sur le plan factoriel défini par les
deux premières composantes (axes 1 et 2), en affichant les noms des
individus et en appliquant la fonction de repulsion pour éviter le
chevauchement des étiquettes.

\begin{Shaded}
\begin{Highlighting}[]
\NormalTok{acp\_resultat }\OtherTok{\textless{}{-}} \FunctionTok{prcomp}\NormalTok{(Data\_centre\_reduit, }\AttributeTok{scale =} \ConstantTok{TRUE}\NormalTok{)}

\CommentTok{\# Tracé du plan factoriel (Axes 1 et 2)}
\FunctionTok{fviz\_pca\_ind}\NormalTok{(acp\_resultat, }\AttributeTok{axes =} \FunctionTok{c}\NormalTok{(}\DecValTok{1}\NormalTok{,}\DecValTok{2}\NormalTok{), }
             \AttributeTok{label =} \StringTok{"ind"}\NormalTok{,  }
             \AttributeTok{repel =} \ConstantTok{TRUE}\NormalTok{,    }
             \AttributeTok{title =} \StringTok{"Projection des individus sur le plan factoriel (Axes 1 et 2)"}\NormalTok{)}
\end{Highlighting}
\end{Shaded}

\includegraphics{Rendu_titouan_files/figure-latex/plan_factoriel-1.pdf}

\begin{Shaded}
\begin{Highlighting}[]
\CommentTok{\# Identifier les individus les plus représentatifs des axes 1 et 2}
\NormalTok{coord\_individus }\OtherTok{\textless{}{-}}\NormalTok{ acp\_resultat}\SpecialCharTok{$}\NormalTok{x}
\NormalTok{ind\_max\_axe1 }\OtherTok{\textless{}{-}} \FunctionTok{rownames}\NormalTok{(coord\_individus)[}\FunctionTok{which.max}\NormalTok{(}\FunctionTok{abs}\NormalTok{(coord\_individus[,}\DecValTok{1}\NormalTok{]))]}
\NormalTok{ind\_max\_axe2 }\OtherTok{\textless{}{-}} \FunctionTok{rownames}\NormalTok{(coord\_individus)[}\FunctionTok{which.max}\NormalTok{(}\FunctionTok{abs}\NormalTok{(coord\_individus[,}\DecValTok{2}\NormalTok{]))]}

\CommentTok{\# Affichage des résultats}
\FunctionTok{cat}\NormalTok{(}\StringTok{"Individu le plus représentatif de l\textquotesingle{}axe 1 :"}\NormalTok{, ind\_max\_axe1, }\StringTok{"}\SpecialCharTok{\textbackslash{}n}\StringTok{"}\NormalTok{)}
\end{Highlighting}
\end{Shaded}

\begin{verbatim}
## Individu le plus représentatif de l'axe 1 : 33
\end{verbatim}

\begin{Shaded}
\begin{Highlighting}[]
\FunctionTok{cat}\NormalTok{(}\StringTok{"Individu le plus représentatif de l\textquotesingle{}axe 2 :"}\NormalTok{, ind\_max\_axe2, }\StringTok{"}\SpecialCharTok{\textbackslash{}n}\StringTok{"}\NormalTok{)}
\end{Highlighting}
\end{Shaded}

\begin{verbatim}
## Individu le plus représentatif de l'axe 2 : 6
\end{verbatim}

\#Question 8 On produit un graphique du cercle des corrélations issu de
l'analyse en composantes principales. On extrait ensuite les
coefficients de chargement associés aux axes principaux pour repérer,
parmi toutes les variables, celles qui influencent le plus le premier et
le deuxième axe. Enfin, on examine le jeu de données standardisé pour
identifier l'observation présentant la valeur la plus extrême pour
chacune de ces variables déterminantes, ce qui permet d'associer les
variables les plus contributives aux individus qui les illustrent le
mieux.

\begin{Shaded}
\begin{Highlighting}[]
\CommentTok{\# Représentation du cercle des corrélations pour les variables}
\FunctionTok{fviz\_pca\_var}\NormalTok{(acp\_resultat, }\AttributeTok{repel =} \ConstantTok{TRUE}\NormalTok{, }\AttributeTok{col.var =} \StringTok{"contrib"}\NormalTok{,}
             \AttributeTok{title =} \StringTok{"Cercle des corrélations"}\NormalTok{)}
\end{Highlighting}
\end{Shaded}

\includegraphics{Rendu_titouan_files/figure-latex/cercle_correlations-1.pdf}

\begin{Shaded}
\begin{Highlighting}[]
\CommentTok{\# Extraction des chargements (loadings) issus de l\textquotesingle{}ACP}
\NormalTok{loadings }\OtherTok{\textless{}{-}}\NormalTok{ acp\_resultat}\SpecialCharTok{$}\NormalTok{rotation}

\CommentTok{\# Identification des variables les plus contributives pour les axes 1 et 2}
\NormalTok{var\_PC1 }\OtherTok{\textless{}{-}} \FunctionTok{names}\NormalTok{(}\FunctionTok{which.max}\NormalTok{(}\FunctionTok{abs}\NormalTok{(loadings[,}\DecValTok{1}\NormalTok{])))}
\NormalTok{var\_PC2 }\OtherTok{\textless{}{-}} \FunctionTok{names}\NormalTok{(}\FunctionTok{which.max}\NormalTok{(}\FunctionTok{abs}\NormalTok{(loadings[,}\DecValTok{2}\NormalTok{])))}
\FunctionTok{cat}\NormalTok{(}\StringTok{"Variable la plus contributive pour PC1 :"}\NormalTok{, var\_PC1, }\StringTok{"}\SpecialCharTok{\textbackslash{}n}\StringTok{"}\NormalTok{)}
\end{Highlighting}
\end{Shaded}

\begin{verbatim}
## Variable la plus contributive pour PC1 : Poids_Total
\end{verbatim}

\begin{Shaded}
\begin{Highlighting}[]
\FunctionTok{cat}\NormalTok{(}\StringTok{"Variable la plus contributive pour PC2 :"}\NormalTok{, var\_PC2, }\StringTok{"}\SpecialCharTok{\textbackslash{}n}\StringTok{"}\NormalTok{)}
\end{Highlighting}
\end{Shaded}

\begin{verbatim}
## Variable la plus contributive pour PC2 : Anneaux
\end{verbatim}

\begin{Shaded}
\begin{Highlighting}[]
\CommentTok{\# Récupération des individus extrêmes pour ces variables à partir du jeu de données standardisé}
\NormalTok{ind\_ext\_PC1 }\OtherTok{\textless{}{-}} \FunctionTok{which.max}\NormalTok{(}\FunctionTok{abs}\NormalTok{(Data\_centre\_reduit[, var\_PC1]))}
\NormalTok{ind\_ext\_PC2 }\OtherTok{\textless{}{-}} \FunctionTok{which.max}\NormalTok{(}\FunctionTok{abs}\NormalTok{(Data\_centre\_reduit[, var\_PC2]))}
\FunctionTok{cat}\NormalTok{(}\StringTok{"Individu extrême pour"}\NormalTok{, var\_PC1, }\StringTok{":"}\NormalTok{, ind\_ext\_PC1, }\StringTok{"}\SpecialCharTok{\textbackslash{}n}\StringTok{"}\NormalTok{)}
\end{Highlighting}
\end{Shaded}

\begin{verbatim}
## Individu extrême pour Poids_Total : 33
\end{verbatim}

\begin{Shaded}
\begin{Highlighting}[]
\FunctionTok{cat}\NormalTok{(}\StringTok{"Individu extrême pour"}\NormalTok{, var\_PC2, }\StringTok{":"}\NormalTok{, ind\_ext\_PC2, }\StringTok{"}\SpecialCharTok{\textbackslash{}n}\StringTok{"}\NormalTok{)}
\end{Highlighting}
\end{Shaded}

\begin{verbatim}
## Individu extrême pour Anneaux : 83
\end{verbatim}

Ainsi, on constate que le poids total et le nombre d'anneaux sur la
coquille sont les variables les plus étroitement liées à l'âge de
l'ormeau. Pour estimer cet âge, il suffirait donc de mesurer ces deux
caractéristiques et de construire un modèle de prédiction. Par ailleurs,
les individus 33 et 83, qui présentent des valeurs extrêmes pour ces
variables, pourraient être considérés comme des outliers et
éventuellement retirés de l'analyse afin de ne pas biaiser les
résultats.

\section{Question 9}\label{question-9}

L'analyse révèle que d'autres variables, telles que le diamètre, la
longueur ou encore le poids décortiqué, peuvent aussi contribuer de
manière significative à la prédiction de l'âge. Même si le poids total
reste la plus importante en dehors des anneaux, ces indicateurs
supplémentaires permettraient d'élaborer un modèle de régression pour
estimer l'âge de l'ormeau. Toutefois, leur pouvoir prédictif étant moins
élevé, la marge d'erreur risque d'être plus importante. Par ailleurs,
certaines mesures, comme le poids décortiqué, nécessitent d'ouvrir
l'ormeau, ce qui peut ne pas être souhaitable pour les chercheurs.

\end{document}
